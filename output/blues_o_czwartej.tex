\beginsong{Czarny blues o czwartej nad ranem}[by={Stare Dobre Małżeństwo}]

\beginverse
Czwarta nad \[A]ranem, może \[cis]sen przyjdzie, \brk \[D]może mnie odwiedz\[A]isz.
Czwarta nad \[E]ranem, może \[fis]sen przyjdzie, \brk \[D]może mni\[E]e odwi\[A]edzisz.
\endverse

\beginverse
Czemu Cię ni\[A]e ma na odległość \[E]ręki?
Czemu m\[fis]ówimy do siebie lis\[cis]tami?
Gdy Ci to śpi\[D]ewam — u mnie pełnia l\[A]ata,
Gdy to usł\[D]yszysz — będzie środek z\[E]imy.
Czemu się b\[A]udzę o czwartej nad r\[E]anem
I włosy T\[fis]woje próbuję ug\[cis]łaskać?
Lecz nigdzie ni\[D]e ma Twoich wł\[A]osów,
Jest tylko bl\[D]ada nocna l\[E]ampka
— łysa śpie\[fis]waczka.
\endverse

\beginverse
Śpiewamy bl\[A]uesa, bo czwarta nad ra\[E]nem,
Tak cicho, ż\[fis]eby nie zbudzić sąs\[cis]iadów.
Czajnik z gw\[D]izdkiem świruje na g\[A]azie,
Myślałby kt\[D]o, że rodem z Manhatt\[E]anu.
\endverse

\beginverse
Czwarta nad \[A]ranem, może \[cis]sen przyjdzie, \brk \[D]może mnie odwiedz\[A]isz.
Czwarta nad \[E]ranem, może \[fis]sen przyjdzie, \brk \[D]może mni\[E]e odwi\[A]edzisz.
\endverse

\beginverse
Herbata cz\[A]arna myśli rozj\[E]aśnia,
A \[fis]list Twój sam się c\[cis]zyta,
Że można go śpi\[D]ewać, \brk za oknem mr\[A]uczą bluesa
Topole z Krupn\[D]icz\[E]ej.
I jeszcze str\[A]ażak wszedł na s\[E]olo,
\[fis]Ten z Mariackiej Wie\[cis]ży.
Jego tr\[D]ąbka, jak księżyc, \brk bi\[A]egnie nad topolą,
Nigdzie się j\[D]ej nie spie\[E]szy.
\endverse

\beginverse
Już pi\[A]ąta, może \[cis]sen przyjdzie, \brk \[D]może mnie odwiedz\[A]isz.
Już pi\[E]ąta, może \[fis]sen przyjdzie, \brk \[D]może mni\[E]e odwi\[A]edzisz.
\endverse

\endsong