\beginsong{Ballada rajdowa}[by={Piosenki Harcerskie}]

\beginverse
Właśnie \[G]tu, na tej ziemi, \brk młody \[D]harcerz meldował
Swą go\[C]towość umierać za \[G]Polskę.
Tak jak \[G]ty niesiesz plecak, \brk on niósł w\[D] ręku karabin,
W sercu \[C]miłość, nadzi\[D]eję i \[G]troskę.
Może tu w \[G]Nowej Słupi, \brk Dale\[D]szycach, Bielinach
Brzo\[C]zowymi krzyżami znacz\[G]one,
Swą dziewczynę\[G] pożegnał, \brk nic nie wie\[D]dząc, że tylko
Kilka \[C]dni mu życia przezn\[G]aczonych.
\endverse

\beginchorus
Naszej \[G]ziemi śpiewajmy, \brk ziemi \[D]pokłon składajmy,
Taki \[C]prosty, ser\[D]deczny, har\[G]cerski.
Niechaj \[G]echo poniesie \brk tę ba\[D]lladę rajdową
W nowe \[C]jutro i \[D]przyszłość \[G]nową.
\endchorus

\beginverse
Na pom\[G]niku wyryto, \brk że szes\[D]naście miał wiosen,
Że był śmi\[C]ały, odważny, rad\[G]osny.
Kiedy pad\[G]ał, płakała \brk cała \[D]puszcza jodłowa.
Nie doc\[C]zekał czekanej tak \[G]wiosny.
I choć \[G]on nie doczekał, \brk to nie \[D]zginął tak sobie,
Przetarł sz\[C]lak, którym dzisiaj wędr\[G]ujesz.
I gdy \[G]tak przy ognisku \brk śpiewasz \[D]sobie balladę,
Tak jak \[C]on w sercu ojczyznę czu\[G]jesz.
\endverse

\beginchorus
Naszej \[G]ziemi śpiewajmy, \brk ziemi \[D]pokłon składajmy,
Taki \[C]prosty, ser\[D]deczny, har\[G]cerski.
Niechaj \[G]echo poniesie \brk tę ba\[D]lladę rajdową
W nowe \[C]jutro i \[D]przyszłość \[G]nową.
\endchorus

\endsong